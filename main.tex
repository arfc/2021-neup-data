\section{Task 1.1: Data Mining and Interface}

The University of Illinois at Urbana-Champaign (UIUC) has a vast amount of
data that can be used to understand patterns and trends at a deep level. Table \ref{tab:datasummary} summarizes all of the data the NEUP team can access.
Before the data from UIUC Facilities and Services (F\&S) can be used, it must
be checked for data flagged as ``unreliable'' or otherwise incorrect values.
The first step in this data processing is shown in Figure \ref{fig:unreliable}
\begin{figure}
  \centering
  \includegraphics[width=\textwidth]{unreliable}
  \caption{Some of the campus data has been flagged as unreliable. This usually caused by instrument failure.}
  \label{fig:unreliable}
\end{figure}
This data has been mined and processed to obtain some interesting relationships.
We've also found some interesting \textit{absent} relationships. Figure \ref{fig:demandfreq} shows two significant peaks in the temperature frequency
curve, and several significant peaks in the demand frequency curve. These peaks
indicate \textit{seasonalities} for each set of data. The peaks at 1 and 365
indicate yearly and daily trends, respectively. However, campus demand exhibits
trends that temperature does not, such as weekly variation due to low demand
on weekends. For this reason, the yearly trends for demand and temperature are
highly correlated, with a Pearson correlation of $r=0.97$, while the daily trends have a much lower correlation of $r=0.69$. Figure \ref{fig:tempcorrelation} shows this correlation and a polynomial characteristic
function that fits the data. This line give a simple first order relationship
that allows us to guess a demand based on the temperature. Figure
\ref{fig:yearlytrends} shows the yearly trends for electricity demand and air
temperature on campus.


\begin{figure}
  \centering
  \includegraphics[width=\textwidth]{demandtempcorr}
  \caption{The correlation and best fit line for campus demand and wet bulb
  temperature.}
  \label{fig:tempcorrelation}
\end{figure}

\begin{figure}
  \centering
  \includegraphics[width=\textwidth]{demandtempfreq}
  \caption{Show the frequency peaks after taking the Fourier transform of data.}
  \label{fig:demandfreq}
\end{figure}

\begin{figure}
  \centering
  \includegraphics[width=\textwidth]{yearlytrends}
  \caption{Show the yearly trends for demand and temperature on campus.}
  \label{fig:yearlytrends}
\end{figure}

\subsection{Characterization: Typical Years}

The energy system has been further characterized by determining \textit{typical years}
on campus. These typical years were generated by identifying the typical month --  ``typical" is defined as the closest to a mean value -- for each month of the
year and combining them to form a typical year. This task was performed using
the \texttt{RAVEN} tool from INL. Figure \ref{fig:typstm} shows the typical year
of steam, Figure \ref{fig:typelc} shows the typical year of electricity demand,
Figure \ref{fig:typwind} shows the typical year of wind power delivered to UIUC, and Figure \ref{fig:typsol} shows the typical year of solar power
delivered to the campus. These typical years are useful for generating synthetic
data, also with the \texttt{RAVEN} tool, that can be used for energy system
optimization with \texttt{Modelica} \cite{epiney_report_2017, baker_optimal_2018}. Figure \ref{fig:synelc} demonstrates this capability.

\begin{figure}
  \centering
  \includegraphics[width=0.8\textwidth]{typicalsteam}
  \caption{A typical year of hourly steam demand on the UIUC campus}
  \label{fig:typstm}
\end{figure}

\begin{figure}
  \centering
  \includegraphics[width=0.8\textwidth]{typicaldemand}
  \caption{A typical year of hourly electricity demand on the UIUC campus}
  \label{fig:typelc}
\end{figure}

\begin{figure}
  \centering
  \includegraphics[width=0.8\textwidth]{typicalwind}
  \caption{A typical year of hourly wind energy supplied to the UIUC campus}
  \label{fig:typwind}
\end{figure}

\begin{figure}
  \centering
  \includegraphics[width=0.8\textwidth]{typicalsolar}
  \caption{A typical year of hourly solar energy supplied to the UIUC campus}
  \label{fig:typsol}
\end{figure}


\begin{figure}
  \centering
  \includegraphics[width=0.8\textwidth]{syntypdemand}
  \caption{A typical year of hourly electricity demand on the UIUC campus}
  \label{fig:synelc}
\end{figure}

\subsection{Other Data}
There is more to the UIUC campus energy system than steam and electricity. UIUC
also has a significant vehicle fleet. Analyzing this data allows us to determine
the demand for gasoline equivalent energy on campus. This demand may be replaced
by either electric or hydrogen powered vehicles in the future. Figure
\ref{fig:fueldemand} shows the demand for three different fuel types on the UIUC
campus.

\begin{figure}
  \centering
  \includegraphics[width=\textwidth]{uiuc_fueldemand}
  \caption{The demand for three types of fuel for one year on the UIUC campus.}
  \label{fig:fueldemand}
\end{figure}
%==============================================================================
%==============================================================================
% BEGIN SUMMARY TABLE
%==============================================================================
%==============================================================================
\begin{landscape}

  \begin{table}
    \centering
    \caption{Summary of Currently Available Data}
    \label{tab:datasummary}
    \begin{tabular}{c|c|c|c|c|c}
      \hline
      Data & Resolution & Span & Supply/Demand & Units & Source\\
      \hline
      Abbott Electricity Generation & Hourly & Fiscal$^{\text{a}}$ Years [2015, 2019]& Supply & kW & UIUC F\&S$^{\text{c}}$\\
      Campus Electricity Demand & Hourly & Fiscal Years [2014, 2019] & Demand & kW & UIUC F\&S \\
      Wind Energy to Campus & Hourly & Fiscal Years [2016, 2019] & Supply & kW & UIUC F\&S \\
      UIUC Solar Farm 1.0 & 15-minute & Calendar Years (2015, 2019] & Supply & kW & AlsoEnergy \cite{alsoenergy_university_2019}\\
      Solar Irradiance & 30-minute & Calendar Years [2013, 2018]& [-] & W/m$^2$& OpenEI \cite{sengupta_national_2018}\\
      Campus Steam Demand & Hourly & Fiscal Years [2015, 2019] & Supply & Klbs & UIUC F\&S\\
      Lincoln Weather Data$^{\text{b}}$ & Hourly & Calendar Years [2010,2019] & [-] & Varied & NOAA \cite{noauthor_climate_nodate} \\
      Champaign Weather Data $^{\text{b}}$& Hourly & Calendar Years [2010,2019]& [-] & Varied & NOAA \cite{noauthor_climate_nodate}\\
      On campus UIUC Fleet Fuel Demand & Daily & Calendar Year [2019]& Demand & Gallons & UIUC F\&S \\
      Total UIUC Fleet Fuel Demand & Yearly & Calendar Year [2019]& Demand & Gallons & UIUC F\&S \\
      CU-MTD Fuel Demand & Daily & Fiscal$^{\text{d}}$ Year [2018]& Demand & Gallons, Dollars & CU-MTD$^{\text{c}}$ \\
      Abbott: Low Pressure Steam & Minute & Calendar Year [2019] & Supply & Klbs & UIUC F\&S\\
      Abbott: High Pressure Steam & Minute & Calendar Year [2019] & Supply & Klbs & UIUC F\&S\\
      Campus Electricity Demand & Minute & Calendar Year [2019] & Demand & kW & UIUC F\&S\\
      Chilled Water System & Minute & Calendar Year [2019] & Supply/Demand & Tons & UIUC F\&S\\
      Thermal Energy Storage & Minute & Calendar Year [2019] & Storage & Tons & UIUC F\&S\\
      UIUC Solar Farm & Minute & Calendar Year [2019] & Supply & kW & UIUC F\&S\\
      UIUC Total Natural Gas & Minute & Calendar Year [2019] & Demand & BTU & UIUC F\&S\\
      Bluewaters Supercomputer & Hourly & Fiscal Years [2014,2018] & Demand & kW & UIUC F\&S\\
    \end{tabular}
    \subcaption{The UIUC fiscal year runs from August 1 to July 31.}
    \subcaption{See Table 2 for further breakdown of weather data.}
    \subcaption{This data is proprietary \textit{unsure about citation}.}
    \subcaption{The CU-MTD fiscal year runs from July 1 to June 30.}
  \end{table}
\end{landscape}

%==============================================================================
%==============================================================================
% BEGIN WEATHER TABLE
%==============================================================================
%==============================================================================

\begin{table}
  \centering
  \caption{Description of available weather data}
  \label{tab:weather}
  \begin{tabular}{c|c}
  \hline
  Variable  & Units\\
  \hline
  Dry Bulb Temp  & $^\circ$F\\
  Wet Bulb Temp  & $^\circ$F\\
  Precipitation  & inches \\
  Relative Humididty  & \%\\
  Wind Direction  & $^\circ$\\
  Wind Speed  & m/s\\
  Station Pressure & in. Hg \\
  \end{tabular}
\end{table}
